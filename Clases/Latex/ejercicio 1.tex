\documentclass[letterpaper, 12pt, oneside]{article}%especificaciones del documento
\usepackage{amsmath}%paquete para escribir expresiones matemámaticas
\usepackage{amssymb}
\usepackage{enumitem}
\usepackage{graphicx}%paquete para poder incluír imagenes en el documento
\usepackage{xcolor} %paquete de LaTex para poder poner otro texto
\graphicspath{{Imagenes/}}%directorio de la imagen, este lo cambian por el directorio en el que ustedes guardaron su imagen 1.png
\usepackage[utf8]{inputenc} %para poder poner acentos

\title{\Huge Taller de Herramientas Computacionales}
%\title{\Huge \colorbox{magenta}{Taller de Herramientas computacionales}} %De esta forma con colorbox pone el texto dentro de una "caja" de color.
\author{Edwin Cardoso Tovar}%autor del escrito
\date{15/01/19}%fecha del escrito

\begin{document}%inicia el documento
	\maketitle
	%\vfill %Para rellenar el espacio y colocar hasta abajo de la pagina el siguiente texto, imagen.
	\begin{center}%inicia centrado
	%	\includegraphics[scale=0.40]{1.png}%del lado izquierdo se muestra el tamaño de la imagen, del derecho se escribe el nombre de la imagen a incluir en el texto
	\end{center}%termina el centrado para la imagen
	\newpage%crea una nueva página
	
	\title{\Huge Clase 1 (Bitácora)  \\}%titulo2 \\ sirven para saltar una linea.
	
Nos hablaron de distintos tipos de sistemas operativos como lo son: Windows, Linux, iOS etc. Y que no todos los archivos se pueden ejecutar en estos sistemas operativos, por ejemplo en Windows existen los archivos .exe y estos no se pueden ejecutar o archivar en cualquier otro sistema. Después el profesor nos hizo mención del "Bash"(Bourne-again shell) el cual es un programa informático, cuya función consiste en interpretar órdenes con un determinado lenguaje, además que no solo es de Linux pues ya se ha llevado a otros sistemas como Windows y Android.\\

Posteriormente nos habló del bit, el bit en informática es la unidad mínima de información que puede tener solo dos valores, cero o uno.\
Hicimos un ejercicio que consistia en saber cuánta información hay 8 bits,qué es la forma en que se desalloraban los videojuegos de antes y como se representa, es decir los 8 bits los podemos representar como 8 casillas y en éstas va a estar el uno o cero dependiendo del estado en que se encuentren (prendidas o apagadas respectivamente) esto nos dio un número de la forma : $\sqrt{2^0} + \sqrt{2^1} + \sqrt{2^1} + \sqrt{2^2} + \sqrt{2^3} + \sqrt{2^4} + \sqrt{2^5} + \sqrt{2^6} + \sqrt{2^7}= 255$






	
\end{document}