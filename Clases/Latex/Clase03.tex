\documentclass[letterpaper, 12pt, oneside]{article}%especificaciones del documento
\usepackage{amsmath}%paquete para escribir expresiones matemámaticas
\usepackage{amssymb}
\usepackage{enumitem}
\usepackage{graphicx}%paquete para poder incluír imagenes en el documento
\usepackage{xcolor} %paquete de LaTex para poder poner otro texto
\graphicspath{{/home/edwincardoso/Descargas/TallerDeLasHerramientasComputacionales/Clases/Latex/Imagenes}}%directorio de la imagen, este lo cambian por el directorio en el que ustedes guardaron su imagen 1.png
\usepackage[utf8]{inputenc} %para poder poner acentos

\title{\Huge Taller de Herramientas Computacionales}
%\title{\Huge \colorbox{magenta}{Taller de Herramientas computacionales}} %De esta forma con colorbox pone el texto dentro de una "caja" de color.
\author{Edwin Cardoso Tovar}%autor del escrito
\date{14/01/19}%fecha del escrito

\begin{document}
	\maketitle
	%\vfill %Para rellenar el espacio y colocar hasta abajo de la pagina el siguiente texto, imagen.
	\begin{center}%inicia centrado
		%\includegraphics[scale=0.40]{1.png}%del lado izquierdo se muestra el tamaño de la imagen, del derecho se escribe el nombre de la imagen a incluir en el texto
	\end{center}%termina el centrado para la imagen
	\newpage%crea una nueva página
	
	\title{\textbf{\Huge Clase 3 (Bitacora)\\}) \\}%titulo2 \\ sirven para saltar una linea.
	
	En esta clase vimos que en Linux hay archivos que no se pueden ver o no se sabe que tipo de archivos son o su información que contienen, así que para saber el tipo de archivo usamos el comando\\
	
	$>file "nombre del archivo"$	(indica el nombre y tipo de archivo)\\
	
	$>ls -la $ (muestra información sobre directorios y archivos\\
	
	si hacemos alguún cambio en algún archivo, va uno al repositorio en el que se encuentra ese archivo y usamos el comando git add* para registrar los cambios , después git commit , seguido de git push y así se actualizan todos los cambios que uno vaya haciendo o si se crea un nuevo archivo desde la computadora.\\
	
	CAT es un comando  que nos muestra el contenido de un archivo y se usa de esta manera:\\
	$>cat "nombre del archivo"$,\\
	tambien aprendimos como hacer nuevos directorios y el comando que los crea, para crear un directorio es de la siguiente manera:\\
	$(nos dirigimos al directorio en el que queremos crear uno nuevo)\\
	>mkdir "nombre del nuevo directorio"$\\
	$>mkdir -p directorio1/direcotio2$ (crea un directorio padre, si se usa mkdir /p crea todos los directorios necesarios para llegar al directorio que estamos creando)\\
	
	Regresando a github aprendimos una forma más rápida de hacer commit , esta es: "git commit -m".\\
	El comando vi crea un documento de texto plano, de hecho vi es un editor de Linux y su comando es vi al momento de crear uno y se usa así:\\
	$>vi clases/latex/03.txt$\\
	$>touch README.md$(crea un documento con el nombre README)\\	
	

	
	
	
	
	
	
	
	
\end{document}
	
	
	
	
	