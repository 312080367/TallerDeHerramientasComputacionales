\documentclass[letterpaper, 12pt, oneside]{article}%especificaciones del documento
\usepackage{amsmath}%paquete para escribir expresiones matemámaticas
\usepackage{amssymb}
\usepackage{enumitem}
\usepackage{graphicx}%paquete para poder incluír imagenes en el documento
\usepackage{xcolor} %paquete de LaTex para poder poner otro texto
\graphicspath{{/home/edwincardoso/Descargas/TallerDeLasHerramientasComputacionales/Clases/Latex/Imagenes}}%directorio de la imagen, este lo cambian por el directorio en el que ustedes guardaron su imagen 1.png
\usepackage[utf8]{inputenc} %para poder poner acentos

\title{\Huge Taller de Herramientas Computacionales}
%\title{\Huge \colorbox{magenta}{Taller de Herramientas computacionales}} %De esta forma con colorbox pone el texto dentro de una "caja" de color.
\author{Edwin Cardoso Tovar}%autor del escrito
\date{14/01/19}%fecha del escrito

\begin{document}%inicia el documento
	\maketitle
	%\vfill %Para rellenar el espacio y colocar hasta abajo de la pagina el siguiente texto, imagen.
	\begin{center}%inicia centrado
		%\includegraphics[scale=0.40]{1.png}%del lado izquierdo se muestra el tamaño de la imagen, del derecho se escribe el nombre de la imagen a incluir en el texto
	\end{center}%termina el centrado para la imagen
	\newpage%crea una nueva página
	
	\title{\textbf{\Huge Clase 6 (Bitacora)\\}) \\}%titulo2 \\ sirven para saltar una linea.
	
	
	\section*{\Huge Python\\} 
	
	Las condiciones ELSE e IF:\\
	Si la condición IF es verdadera se ejecuta sin bloque, en caso de que la condición sea falsa se ejecuta el bloque de ELSE. En cualquier de los dos casos la instrucción sólo se ejecuta una vez, el cilo de ELSE e IF se escriben:
	
	\textbf{if "la condición":}\ 
	
	\textbf{"la instruccion"}\\
	\textbf{else:}\\
	\textbf{"la instrucción"}\\
	es importante no olvidar los dos puntos (:)sino nos marcará un error.
	Un bloque son las instrucciones que están identadas a partir de un cierto nivel, o sea todo lo que va después de dos puntos(:).Aquí podemos colocar varias funciones if como sean necesarias.
	
	Un ALGORITMO es un proceso de instrucciones finitas.\\
	
	Una ASIGNACIÓN es distinta de una IGUALDAD en Python, la asignación consiste en "asignar" valores directamente  a través de una fórmula y también a través de una letra o número. En Python diferenciamos la asiganción de la igualdad de la siguiente manera:
	a=4 (esto es una asignación)
	a==4 esto es una igualdad o semejanza.\\
	
	Los operadores lógicos por otra parte se utilizan para juntar o concatenar condicones que nosotros queramos poner, estos operadores lógicos son los siguientes:\\
	
	and: que es verdadero si todas las condiciones que une son verdaderas\\
	or: es verdadero si al menos una de las condiciones que une son verdaderas.\\
	
	WHILE es utilizaod para que se ejecute una instruccion o serie de instrucciones en una cierta cantidad de veces. Este ciclo se escribe:\\
	
	while "la condicion que queramos poner":\\
	"la instruccion"
	
	
	
		
	
	
	
	
	
	
	
	


	
	
	
	
	
	
	
	
	
	
	
	
	
	
	
	
\end{document}