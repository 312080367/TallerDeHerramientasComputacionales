\documentclass[letterpaper, 12pt, oneside]{article}%especificaciones del documento
\usepackage{amsmath}%paquete para escribir expresiones matemámaticas
\usepackage{amssymb}
\usepackage{enumitem}
\usepackage{graphicx}%paquete para poder incluír imagenes en el documento
\usepackage{xcolor} %paquete de LaTex para poder poner otro texto
\graphicspath{{/home/edwincardoso/Descargas/TallerDeLasHerramientasComputacionales/Clases/Latex/Imagenes}}%directorio de la imagen, este lo cambian por el directorio en el que ustedes guardaron su imagen 1.png
\usepackage[utf8]{inputenc} %para poder poner acentos

\title{\Huge Taller de Herramientas Computacionales}
%\title{\Huge \colorbox{magenta}{Taller de Herramientas computacionales}} %De esta forma con colorbox pone el texto dentro de una "caja" de color.
\author{Edwin Cardoso Tovar}%autor del escrito
\date{14/01/19}%fecha del escrito

\begin{document}
	\maketitle
	%\vfill %Para rellenar el espacio y colocar hasta abajo de la pagina el siguiente texto, imagen.
	\begin{center}%inicia centrado
		%\includegraphics[scale=0.40]{1.png}%del lado izquierdo se muestra el tamaño de la imagen, del derecho se escribe el nombre de la imagen a incluir en el texto
	\end{center}%termina el centrado para la imagen
	\newpage%crea una nueva página
	
	\title{\textbf{\Huge Clase 5 (Bitacora)\\} \\}%titulo2 \\ sirven para saltar una linea.
	
	\section*{PYTHON}\\
	
	CADENAS DE CARACTERES\\
	'...': una comilla para cadena de texto de una linea\\
	"....": doble comilla , texto de una linea\\
	'''....''':tres comillas para una cadena de texto multilinea\\
	$>"barra invertida"n $ para salto de linea, por ejemplo:\\
	$> esta "barra invertida"n es una "barra invertida"n cadena$\\
	Una cadena en Python es un conjunto de características que están delimitadas comillas.
	El comando PRINT hace eco de la orden que le damos .
	$>(simbolo porcentaje)g$ muestra la variable en el formato numerico mas corto posible\\
	$>(simbolo porcentaje)E$ lo muestra en formato de notacion cientifica
	$>(simbolo porcentaje).2f$ muestra un valor flotante con 2 decimales y puesde ser cualquier numero de decimales\\
	$>(simbolo porcentaje)10.2f$nos muestra un numero flotante con dos decimales recorrido diez espacios a la derecha\\
	$>(simbolo porcentaje)f$nos da el valor en formato flotante\\
	
	
	
	
	
	

	
	
	
	
	
	
	
	
\end{document}
	
	