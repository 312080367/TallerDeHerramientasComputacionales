\documentclass[letterpaper, 12pt, oneside]{article}%especificaciones del documento
\usepackage{amsmath}%paquete para escribir expresiones matemámaticas
\usepackage{amssymb}
\usepackage{enumitem}
\usepackage{graphicx}%paquete para poder incluír imagenes en el documento
\usepackage{xcolor} %paquete de LaTex para poder poner otro texto
\graphicspath{{/home/edwincardoso/Descargas/TallerDeLasHerramientasComputacionales/Clases/Latex/Imagenes}}%directorio de la imagen, este lo cambian por el directorio en el que ustedes guardaron su imagen 1.png
\usepackage[utf8]{inputenc} %para poder poner acentos

\title{\Huge Taller de Herramientas Computacionales}
%\title{\Huge \colorbox{magenta}{Taller de Herramientas computacionales}} %De esta forma con colorbox pone el texto dentro de una "caja" de color.
\author{Edwin Cardoso Tovar}%autor del escrito
\date{14/01/19}%fecha del escrito

\begin{document}
	\maketitle
	%\vfill %Para rellenar el espacio y colocar hasta abajo de la pagina el siguiente texto, imagen.
	\begin{center}%inicia centrado
		%\includegraphics[scale=0.40]{1.png}%del lado izquierdo se muestra el tamaño de la imagen, del derecho se escribe el nombre de la imagen a incluir en el texto
	\end{center}%termina el centrado para la imagen
	\newpage%crea una nueva página
	
	\title{\textbf{\Huge Clase 2 (Bitacora)\\}) \\}%titulo2 \\ sirven para saltar una linea.
	
	El profesor nos hizo la recomendación de installar una máquina virtual en equipos con 4 o más cores(núcleos), repasamos el comando top que nos muestra la información sobre la coputadora como el CPU y su actividad.
	También vimos el comando set que muestra las vaiables de entorno y un nuevo comando que es "set|less", este comando lo que hace es paginar las variables de entorno pues "set" las muestr por lista.
	Además de eso vimos otros 3 nuevos comandos que son :\\
	$>less (muestra el contenido de un archivo paginado)$\\
	$>file (Despliega los archivos de otro archivo o de un directorio)$\\
	$>cd  /dev (aparatos o partes conectadas con el equipo)$\\
	
	Posteriormente vimos lo que es git y uno de sus servidores github, git es un software que está pensado para  grupos de desarrolldores o programadores.
	git hub es uno de sus servidores que nos sirve para bajar información , subir información, registrar cambios o regresar cambios y lo vamos a estar utilizando para subir nuestras bitacoras y programas .
    Nos enseñaron como instalar github y como crear un repositorio, a continuación pondré los comandos:\\
    (para instalar github)\\
    $>git init$\\
    $>sudo apt-get upgrade$  (en fedora es dnf en lugar de apt)\\
    $>sudo apt-get install git$
    $>git config --global user.email "asaped96@gmail.com"$\\
    $>git config --global user.name "312080367"$\\
    
    creamos un repositorio thc y compiamos el enlace de nuestro repositorio :https://github.com/312080367/TallerDeHerramientasComputacionales.git\\
    Algo importante que debemos saber es que solo se clona el repositorio cuando por primera vez vamos a descargar nuestros archivos a nuestra computadora.
    A continuacion les dare las instrucciones de como crear un repositorio:
    En git hub creamos nuestro repositorio , copiamos nuestro enlace (es el que tengo arriba) y usar los siguentes comandos en el bash:\\
    $>cd Descargas (este es el directorio en el que vas a crear el repositorio)$\\
    $>git clone "aqui pegas tu enlace del repositorio"$\\
    $>git cd "nombre del repositorio"$\\
    $>git add *$\\
    $>git commit (este comando es para hacer un comentario en el repositorio, para salirse del commit presinas las teclas esc + shift + : . si hiciste un comentario escribes wq y si no hiciste nada solo q)$\\ 
    $>git push $\\
    $>escribes tu nombre de usuario$\\
    $>escribes tu password$\\
  
	
	
	
	
	
	
	
	
	
	
	


\end{document}