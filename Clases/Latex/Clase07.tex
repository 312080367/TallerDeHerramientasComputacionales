\documentclass[letterpaper, 12pt, oneside]{article}%especificaciones del documento
\usepackage{amsmath}%paquete para escribir expresiones matemámaticas
\usepackage{amssymb}
\usepackage{enumitem}
\usepackage{graphicx}%paquete para poder incluír imagenes en el documento
\usepackage{xcolor} %paquete de LaTex para poder poner otro texto
\graphicspath{{/home/edwincardoso/Descargas/TallerDeLasHerramientasComputacionales/Clases/Latex/Imagenes}}%directorio de la imagen, este lo cambian por el directorio en el que ustedes guardaron su imagen 1.png
\usepackage[utf8]{inputenc} %para poder poner acentos

\title{\Huge Taller de Herramientas Computacionales}
%\title{\Huge \colorbox{magenta}{Taller de Herramientas computacionales}} %De esta forma con colorbox pone el texto dentro de una "caja" de color.
\author{Edwin Cardoso Tovar}%autor del escrito
\date{15/01/19}%fecha del escrito

\begin{document}%inicia el documento
	\maketitle
	%\vfill %Para rellenar el espacio y colocar hasta abajo de la pagina el siguiente texto, imagen.
	\begin{center}%inicia centrado
		%\includegraphics[scale=0.40]{1.png}%del lado izquierdo se muestra el tamaño de la imagen, del derecho se escribe el nombre de la imagen a incluir en el texto
	\end{center}%termina el centrado para la imagen
	\newpage%crea una nueva página
	
	\title{\Huge Clase 7 (Bitácora)  \\}%titulo2 \\ sirven para saltar una linea.
	
En esta clase se hizo mucho hincapié sobre el comando WHILE en Python, este comando representa un ciclo y funciona :"Mientras algo sea verdadero", por eso la función de While es repetir instrucciones un cierto número de veces y esto depende de la condición que se defina en un principio.Para ver el resultado de cada ejecución es necesario poner un comando PRINT identado al ciclo WHILE , ahora si si PRINT no está incluido en el ciclo el resultado lo escribe al final. También aprendimos que para contar cuantas veces se repite WHILE se agrega una variable o comando (no esto muy seguro qué es, debo preguntar) i=0 y dentro del ciclo WHILE escribir i=1+1.\\

Esto nos llevó a algo muy importante, tantos mis compañeros como yo tuvimos ciertos errores y esto fue que no dejabamos espacios al definir bloques cosa que es muy importante al momento de hacer un programa, de igual forma vimos que al definifir una solución a un problema con Python era importante saber las partes que tiene una función como lo son:condiciones iniciales, arrojar valores, cálculos, etc.\\
Siempre es mejor hacer un programa por partes ya que así es más fácil hacerle correcciones.
Tamibién vimos que el operador módulo en Python es un signo de porcentaje el cual nos ayudo y sirvió para identificar una divisióne entera , esto nos ayudó a resolver nuestro problema.\\

Desués pasamos a LaTex donde aprendimos como "matemáticamente" esto va entre dos signos de peso y ahí es donde escribimos sumas, potencias, raíces ,escuaciones, integrales etc. También vimos que con el comando SECTION hacemos un encabezado numerado y si ponemos SECTION* no lo enumera.

	
	
	
	
	
	
	
	
	
	
	
	
	
	
	
	
\end{document}
