\documentclass[letterpaper, 12pt, oneside]{article}%especificaciones del documento
\usepackage{amsmath}%paquete para escribir expresiones matemámaticas
\usepackage{amssymb}
\usepackage{enumitem}
\usepackage{graphicx}%paquete para poder incluír imagenes en el documento
\usepackage{xcolor} %paquete de LaTex para poder poner otro texto
\graphicspath{{/home/edwincardoso/Descargas/TallerDeLasHerramientasComputacionales/Clases/Latex/Imagenes}}%directorio de la imagen, este lo cambian por el directorio en el que ustedes guardaron su imagen 1.png
\usepackage[utf8]{inputenc} %para poder poner acentos

\title{\Huge Taller de Herramientas Computacionales}
%\title{\Huge \colorbox{magenta}{Taller de Herramientas computacionales}} %De esta forma con colorbox pone el texto dentro de una "caja" de color.
\author{Edwin Cardoso Tovar}%autor del escrito
\date{07/01/19}%fecha del escrito

\begin{document}%inicia el documento
	\maketitle
	%\vfill %Para rellenar el espacio y colocar hasta abajo de la pagina el siguiente texto, imagen.
	\begin{center}%inicia centrado
		%\includegraphics[scale=0.40]{1.png}%del lado izquierdo se muestra el tamaño de la imagen, del derecho se escribe el nombre de la imagen a incluir en el texto
	\end{center}%termina el centrado para la imagen
	\newpage%crea una nueva página
	
	\title{\Huge Clase 1 (Bitácora)  \\}%titulo2 \\ sirven para saltar una linea.
	
	Nos hablaron de distintos tipos de sistemas operativos como lo son: Windows, Linux, iOS etc. Y que no todos los archivos se pueden ejecutar en estos sistemas operativos, por ejemplo en Windows existen los archivos .exe y estos no se pueden ejecutar o archivar en cualquier otro sistema. Después el profesor nos hizo mención del "Bash"(Bourne-again shell) el cual es un programa informático, cuya función consiste en interpretar órdenes con un determinado lenguaje, además que no solo es de Linux pues ya se ha llevado a otros sistemas como Windows y Android.\\
	
	Posteriormente nos habló del bit, el bit en informática es la unidad mínima de información que puede tener solo dos valores, cero o uno.\
	Hicimos un ejercicio que consistia en saber cuánta información hay 8 bits,qué es la forma en que se desalloraban los videojuegos de antes y como se representa, es decir los 8 bits los podemos representar como 8 casillas y en éstas va a estar el uno o cero dependiendo del estado en que se encuentren (prendidas o apagadas respectivamente) esto nos dio un número de la forma :  $2^0 + 2^1 + 2^2 + 2^3 + 2^4 + 2^5 + 2^6 + 2^7= 255$ que es cuando todas las casillas se encuentran con el número uno, de igual forma nos enseñó los tres tipos de permisos orientado a tres objetos, que son: usuario, grupo y todos. Estos tres permisos son lectura, escritura  y ejecución ahora imaginemos que tenemos un archivo  en nuestro ordenador y estos permisos están configurados de la siguiente manera:\\
	USUARIO: rwx\\
	GRUPO:rw-\\
	TODOS:r--      Donde r=4, w=2 y x=1.\\
	
	Para ver los permisos en la terminal se usan los siguientes comandos:\\
	$>ls -l /bin/bash$\\
	$>touch  /tmp/algo$\\
	$>ls -l   /tmplalgo$\\
	$>chmod   700   /tmp/algo$\\
	$>chmod  751   /tmp/algo$\\
	$>/tmp/algo$\\
	
	
	en estos comandos lo que es importante recalcar son los valores después de chmod ,aquí lo que debemos de ver es que en 700 cada dígito representa a usuario, grupo y todos respectivamente. el número del usuario es el 7 y este tiene el valor de 7 pues los permiso que son rwx , si sumamos sus valores dan un total de 7 por lo tanto tiene todos los permisos, de forma análoga seria con el 751 dónde el usuario tiene todos los permios, el grupo leer y ejecutar y todos sólo ejecutar.\\
	
	Siguiendo con la clase aprendimos lo que es una variable de entorno , una variable de entorno es información que define un comportamiento y tanto Windows como Linux tienen sus variables de entorno pero ambos utilizan utilizan PATH,esta variable nos permite saber las rutas que están definidas en mi sistema Linux y para eso debemos usar los siguientes comandos:\\
	$>echo (signo de pesos)PATH$\\
	$>set (para observar las variables de entorno)$\\
	$>pwd (indica el direcotrio en el que me encuentro)$\\
	$>cd /tmp$\\
	$.  /algo$\\
	
	
	
	
	
	
	
	
	
	
	
	
	
	
	
	
\end{document}
