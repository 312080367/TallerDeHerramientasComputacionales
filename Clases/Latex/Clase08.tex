\documentclass[letterpaper, 12pt, oneside]{article}%especificaciones del documento
\usepackage{amsmath}%paquete para escribir expresiones matemámaticas
\usepackage{amssymb}
\usepackage{enumitem}
\usepackage{graphicx}%paquete para poder incluír imagenes en el documento
\usepackage{xcolor} %paquete de LaTex para poder poner otro texto
\graphicspath{{/home/edwincardoso/Descargas/TallerDeLasHerramientasComputacionales/Clases/Latex/Imagenes}}%directorio de la imagen, este lo cambian por el directorio en el que ustedes guardaron su imagen 1.png
\usepackage[utf8]{inputenc} %para poder poner acentos

\title{\Huge Taller de Herramientas Computacionales}
%\title{\Huge \colorbox{magenta}{Taller de Herramientas computacionales}} %De esta forma con colorbox pone el texto dentro de una "caja" de color.
\author{Edwin Cardoso Tovar}%autor del escrito
\date{16/01/19}%fecha del escrito

\begin{document}
	\maketitle
	%\vfill %Para rellenar el espacio y colocar hasta abajo de la pagina el siguiente texto, imagen.
	\begin{center}%inicia centrado
		%\includegraphics[scale=0.40]{1.png}%del lado izquierdo se muestra el tamaño de la imagen, del derecho se escribe el nombre de la imagen a incluir en el texto
	\end{center}%termina el centrado para la imagen
	\newpage%crea una nueva página
	
	\title{\textbf{\Huge Clase 8 (Bitacora)\\} \\}%titulo2 \\ sirven para saltar una linea.
	
	En esta clase aprendimos mas comandos y funciones de bash. Por ejemplo si estamos en bash y abrimos cualquier programa por ejemplo TexStudio al instante no podemos seguir dando comandos, entonces lo que hacemos es presionar CTRL + Z , con esto saldremos inmediato de la aplicación poniendola en pausa.El problema aquí es que no vamos a poder seguir trabajando en Texstudio  y para eso en la termina escribimos bg, con bg mantenemos el programa pausado en segundo plano y con fg en primer plano.Si lo que queremos es cerrar o "matar" el programa de una vez entonces usaremos el comando CTRL + C y listo, el programa se ha cerrado.
	Otra forma de cerrar el programa es escribiendo en la terminal: kill -9 "PID del programa" el PID son unos números únicos de identificación de cualquier programa que queramos cerrar, es muy importante que no se nos olvide el "-9" después de "kill  pues este comando por sí sólo no hace nada.
	Hicimos un repaso del comando "top" y aprendí que después de poner este comando si precionamos 1 nos mostrará los núcleos o cores.\\
	
	Posteriormente  seguimos viendo comandos en bash pero ahora  enfocandonos más en python; creamos un programa con el nombre 08.py y nos dirigmos al bash y usamos los siguientes comandos:\\
	$ >ls -l 08.py$estando en el direcotrio dónde pusimos el programa, este comando nos muestra las caracteristicas del programa como hora y fecha de creación.\\
	$>chmod +x 08.es$este sirve para darle los permios rwx, en este caso sólo se puede ejecutar pero si quisieramos que tenga todos los permisos se pone rwx después del + .\\
	$> ./08.py$ ejecuta el programa de python en el shell.\\
	$> python 08.py $es otra menera de ejecutar el programa.\\
	$> find . -name "*py"$ nos muestra todos los archivos del tipo .py.\\
	$> whereis python$ nos muestra el lugar donde se encuentra el programa Python.\\
	
	Esto es en idle:\\
	 (almohadilla)!/usr/bin/python2.7 \\en nuestro idle al ponerlo en la parte superior le indica a bash que es un programa de Python.\\
 (almohadilla)(quión bajo)*(guión bajo)coding: utf/8(guión bajo)*(guión bajo)\\ esto para que al escribir acentos,ñ,u otros carácteres propios del español no nos marque un error al correr el programa.\\	
 Algo muy importante es saber diferenciar cuando hay una cadena y una variable por ejemplo: "x,y,z"= es una cadena y x,y,z=es una variable.
	
	
	
	

	
	
	
	
	
	
	
	
\end{document}
	