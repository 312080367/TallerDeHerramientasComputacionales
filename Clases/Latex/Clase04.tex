\documentclass[letterpaper, 12pt, oneside]{article}%especificaciones del documento
\usepackage{amsmath}%paquete para escribir expresiones matemámaticas
\usepackage{amssymb}
\usepackage{enumitem}
\usepackage{graphicx}%paquete para poder incluír imagenes en el documento
\usepackage{xcolor} %paquete de LaTex para poder poner otro texto
\graphicspath{{/home/edwincardoso/Descargas/TallerDeLasHerramientasComputacionales/Clases/Latex/Imagenes}}%directorio de la imagen, este lo cambian por el directorio en el que ustedes guardaron su imagen 1.png
\usepackage[utf8]{inputenc} %para poder poner acentos

\title{\Huge Taller de Herramientas Computacionales}
%\title{\Huge \colorbox{magenta}{Taller de Herramientas computacionales}} %De esta forma con colorbox pone el texto dentro de una "caja" de color.
\author{Edwin Cardoso Tovar}%autor del escrito
\date{14/01/19}%fecha del escrito

\begin{document}
	\maketitle
	%\vfill %Para rellenar el espacio y colocar hasta abajo de la pagina el siguiente texto, imagen.
	\begin{center}%inicia centrado
		%\includegraphics[scale=0.40]{1.png}%del lado izquierdo se muestra el tamaño de la imagen, del derecho se escribe el nombre de la imagen a incluir en el texto
	\end{center}%termina el centrado para la imagen
	\newpage%crea una nueva página
	
	\title{\textbf{\Huge Clase 6 (Bitacora)\\} \\}%titulo2 \\ sirven para saltar una linea.
	
	Empezamos con una pequeña introdución a Python, se nos recomendó que usaramos la versión 2.7.15 en lugar de la versión 3.6 pues está última aún está en desarrollo y por lo tanto puede tener alguna falla y además todos los modulos paquetes están ḿas desarrollados en la versión 2.7.15.
	Aprendimos a abrir Python desde la terminal y lo que es un IDE.Un IDE es un entorno de desarrollo integrado y nos ayuda a programar en Python este IDE tiene un editor en Python que es idle este idle es un interprete de comandos.\\
	
	Vimos los pasos a seguir para resolver un problema, primero tenemos que DEFINIR el problema y esto es entenderlo, si no lo entendemos no seremos capaces de llegar a una solución.Posteriormente lo ANALIZAMOS Y DELIMITAMOS para así llegar  a la SOLUCIÓN.\\
	
	El problema era saber la posicion de una pelota con respecto al tiempo, aqui usamos distintas variables como : $y=posicion,v_{0}=velocidad inicial,t=tiempo,g=gravedad(9.81,esta es una constante)$\\
	La ecuacion que utilizamos fue:\\
	$y(t)=v_{0}*t + 1.0/2*g*t^{2}$
	
	
	
	
	
	
	

	
	
	
	
	
	
	
	
\end{document}
	