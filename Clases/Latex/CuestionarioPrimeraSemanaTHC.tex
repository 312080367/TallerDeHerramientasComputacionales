\documentclass[letterpaper, 12pt, oneside]{article}%especificaciones del documento
\usepackage{amsmath}%paquete para escribir expresiones matemámaticas
\usepackage{graphicx}%paquete para poder incluír imagenes en el documento
\usepackage{xcolor} %paquete de LaTex para poder poner otro texto
\graphicspath{{Imagenes/}}%directorio de la imagen, este lo cambian por el directorio en el que ustedes guardaron su imagen 1.png
\usepackage[utf8]{inputenc} %para poder poner acentos

	\title{\Huge Taller de Herramientas Computacionales}
	%\title{\Huge \colorbox{magenta}{Taller de Herramientas computacionales}} %De esta forma con colorbox pone el texto dentro de una "caja" de color.
	\author{Edwin Cardoso Tovar}%autor del escrito
	\date{13/01/19}%fecha del escrito

\begin{document}%inicia el documento
\maketitle
%\vfill %Para rellenar el espacio y colocar hasta abajo de la pagina el siguiente texto, imagen.
\begin{center}%inicia centrado
\includegraphics[scale=0.40]{1.png}%del lado izquierdo se muestra el tamaño de la imagen, del derecho se escribe el nombre de la imagen a incluir en el texto
\end{center}%termina el centrado para la imagen
\newpage%crea una nueva página

\title{\Huge Cuestionario Primera Semana THC\\}%titulo2 \\ sirven para saltar una linea.


\begin{enumerate}%Inicio de númeración para enlistar las cosas vistas en clase.
	\item ¿Qué es Linux?\\ 
LINUX o GNU/LINUX,es un Sistema Operativo como MacOS, DOS o Windows. Es decir, Linux es el software necesario para que tu ordenador te permita utilizar programas como: editores de texto, juegos, navegadores de Internet, etc. Linux puede usarse mediante un interfaz gráfico al igual que Windows o MacOS, pero también puede usarse mediante línea de comandos como DOS.: %item sirve enlistar el elemento, este es el primer elemento enumerado.
	
	\item ¿Qué es el núcleo de Linux?\\
El núcleo o Kernel de Linux es un conjunto de drivers necesarios para usar el ordenador y haya una conexión entre software y hardware. %Segundo elemento enumerado.
	
	\item ¿Cuáles son algunas de las distintas distribuciones más populares de Linux?\\
De las distribuciones más populares de Linux están: Debian, Fedora,Ubuntu,Slax , etc.
 
     \item Mencione ejemplos de shell:\\
BASH en Linux, PowerShell en Windows
     
     \item Mencione ejemplos de shell\\ 
     BASH en Linux, PowerShell en Windows

     
     \item ¿Para qué sirve el comando TOUCH?\\
Sirve para crear un fichero vacio, y si ya existe uno le modifica le modifica la hora y fecha.
  
     \item ¿Para qué sirve el comando PWD?\\
Este comando nos ayuda a comprobar en qué directorio nos encontramos

    \item ¿cuál es la función del comando LS?\\
Para ver los archivos de un directorio
    
    \item¿Con qué comando uno puedo ver el kernel del sistema operativo?\\
uname -a

    \item ¿Qué es una variable de entorno?\\
Es iformació que define un comportamieto.

    \item ¿Para qué sirve la variable de entorno PATH?\\
Nos muestra la ruta o camino de busqueda de los archivos binarios

    \item ¿Qué fucnión tiene el comando LESS?\\
Permite revisar el contenido de un archivo

    \item ¿Cuál es la utilidad  que nos permite ejecutar comandos con privilegios de administrador?\\
SUDO

    \item ¿Qué es Git?\\
git es un software de control de versiones pensando y diseñado en la eficiencia y la confiabilidad del mantenimiento de versiones de aplicaciones cuando éstas tienen un gran número de archivos de código fuente. 
Su propósito es llevar registro de los cambios en archivos de computadora y coordinar el trabajo que varias personas realizan sobre archivos compartidos.
    
    \item ¿Cómo se crea un repositorio nuevo de git?\\
  git init 

    \item ¿Para qué sirve el comando >git push?\\
Para subir un repositorio a git

    \item ¿Cuales son los comandos para instalar git si no está en tu ordenador?\\
sudo apt-get update
sudo apt-get install git
 
    \item ¿Cuáles son los comandos para añadir un nuevo repositorio git?\\
git add *
git commit (se agrega el comentario , presiona uno las teclas esc + shift + :.
Si no se escribio comentario escribimos:  
Si se escribió un comentario: wq
(posteriormente usamos el comando git push)
git push ( instroducimos nuestro nombre de usaurio y contraseña) y listo .

    
    \item ¿Qué es vi en Fedora?\\
Es un editor desde la consola de Linux por lo cual solo se utiliza el teclado; podrás crear y modificar archivos
de texto .

    \item ¿Cuál es el comando para crear un archivo de texto?\\
 vi "nombre del archivo" .md . ejemplo: vi Readme.md
 
    \item ¿Como editar un archivo con vi?\\
 vi Readme.md, presionar la tecla "i" e insertar las modificaciones que uno le quiere hacer.
 
    \item ¿Cuál es el comando que me permite mostrar archivos de texto en la terminal?\\
cat , ejemplo: cat README.md
     
    \item ¿Cómo salirse del editor vi?\\
Presionar las teclas esc + shift + :
Si no se modificó nada presionar "q"
Si se hizo alguna modificación presionar "wq"
Si se hizo algún cambio pero no lo queremos presionar "q!"

    \item ¿Para qúe nos sirve el comando mkdir?\\ 
Para crear un directorio

    \item ¿Cuál es la función de mkdir -p?\\
Funcina de manera que crea más subdirectorios o directorios padre

    \item ¿Qué es Python?\\
Python es un lenguaje de scripting independiente de plataforma y orientado a objetos, preparado para realizar cualquier tipo de programa, desde aplicaciones Windows a servidores de red o incluso, páginas web.
 Es un lenguaje interpretado, lo que significa que no se necesita compilar el código fuente para poder ejecutarlo, lo que ofrece ventajas como la rapidez de desarrollo e inconvenientes como una menor velocidad.

    \item ¿Qué es un IDE?\\
Un entorno de desarrollo integrado es una aplicación informática que proporciona servicios integrales
para facilitarle al desarrollador o programador el desarrollo de software.

    \item ¿Cúal es el IDE para Python?\\
IDLE (Integrated DeveLopment Environment for Python) es el entorno de desarrollo que permite
editar y ejecutar los programas.

    \item¿Cómo activar IDLE desde Python?\\
Escribir idle y presionar enter.

    \item ¿Qué hace el comando PRINT en Python?\\
Hace eco de la instrucción dada.

    \item ¿Cómo mantener en segundo Plano a Python si se necesitan hacer otros comandos desde bash?\\
con el comando bg

    \item ¿Qué es una cadena en Python?\\
es algo que está encerrado entre comillas, ejemplo: 'esto es una cadena'

    \item ¿Cómo escribir un texto multilinea?\\
    Con tres comillas '''....''' . Ejemplo: ''' para un texto multilinea '''
    
    \item ¿Qué es LaTex?\\
 es un sistema de composición de textos, orientado a la creación de documentos escritos que presenten una alta calidad tipográfica. 
Por sus características y posibilidades, es usado de forma especialmente intensa en la generación de artículos y libros científicos que incluyen, entre otros elementos, expresiones matemáticas.

    \item ¿Qué es TexStudio?\\
TeXstudio es un editor de LaTeX de código abierto y Multiplataforma con una interfaz similar a Texmaker.
TeXstudio es un IDE de LaTeX que proporciona un soporte moderno de escritura, como la corrección ortográfica interactiva, plegado de código y resaltado de sintaxis.
Originalmente llamado TexMakerX, TeXstudio se inició como un Fork de Texmaker que trató de extenderlo con características adicionales, manteniendo su apariencia. 
Se ejecuta en Windows, Unix / Linux, BSD, y sistemas Mac OS X.

    
    


\end{enumerate}%finaliza el enlistado principal


\end{document}%termina el documento
